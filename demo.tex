\documentclass[12pt,a4paper]{article} % Define o tamanho da fonte e do papel

% Pacotes necessários
\usepackage[brazil]{babel} % Define o idioma do documento
\usepackage[utf8]{inputenc} % Permite digitar caracteres especiais
\usepackage{graphicx} % Permite inserir imagens
\usepackage{geometry} % Permite ajustar as margens
\usepackage{titling} % Permite personalizar o título e o autor
\usepackage{fancyhdr} % Permite personalizar o cabeçalho e o rodapé
\usepackage{hyperref} % Permite criar hyperlinks
\usepackage{cite} % Permite citar referências
\usepackage{textcase} % Permite usar \MakeUppercase
\usepackage{setspace} % Permite definir espaçamento entre linhas
\usepackage{indentfirst} % Indenta o primeiro parágrafo de cada seção

% Configuração de margens conforme ABNT
\geometry{
  a4paper,
  left=3cm,
  right=2cm,
  top=3cm,
  bottom=2cm
}

% Configurações de parágrafo e espaçamento
\setlength{\parindent}{1.25cm} % Define o recuo de parágrafo
\setlength{\parskip}{0cm} % Define o espaçamento entre parágrafos
\onehalfspacing % Define o espaçamento 1,5

% Título e autores
\title{\textbf{\MakeUppercase{Relatório Técnico-Científico}}}
\author{Bruna Beatriz, Éber Nascimento, Edivaldo Coelho, Fabyane Nayara, Richard Henrique, Ruan Thiago, Silas Sousa e Thyago Henrique}
\date{Recife/PE \\ 2024.1}

% Configuração de cabeçalho e rodapé
\pagestyle{fancy}
\fancyhf{}
\fancyfoot[C]{\thepage}

% Definindo um novo comando para \maketitle conforme ABNT
\renewcommand{\maketitle}{\begin{titlepage}
\begin{center}
\includegraphics[width=5cm]{img/Unicap_Icam_Tech-01.png} \\ % Insere a logo da UCP
\vspace*{1cm}

\textbf{\Large\scshape UNIVERSIDADE CATÓLICA DE PERNAMBUCO} \\
\vspace*{0.5cm}
\textbf{\Large\scshape CURSO DE SISTEMAS PARA INTERNET} \\
\vspace*{2cm}

\textbf{\fontsize{18pt}{\baselineskip}\selectfont \MakeUppercase{\thetitle}} \\

\vspace*{0.5cm}

\textbf{\fontsize{14pt}{\baselineskip}\selectfont \MakeUppercase{Chatbot Protocolos Acadêmicos}} \\

\vspace*{2cm}

\textbf{\fontsize{14pt}{\baselineskip}\selectfont \MakeUppercase{\theauthor}} \\

\vfill

\textbf{\Large\MakeUppercase{\thedate}}

\end{center}
\end{titlepage}}

\begin{document}

\maketitle

\newpage

\begin{abstract}
\noindent Este relatório técnico-científico descreve o desenvolvimento e implementação do Projeto Chatbot Protocolos Acadêmicos, realizado para a secretaria da Unicap Icam Tech com o intuito de contribuir e colaborar com a melhoria do atendimento dos protocolos acadêmicos. O objetivo principal do projeto é facilitar o atendimento aos alunos e o processamento de solicitações por parte dos funcionários da Universidade Católica de Pernambuco (Unicap). Neste documento, são apresentados os fundamentos teóricos, as ferramentas e tecnologias utilizadas, a metodologia, o desenvolvimento do chatbot e do servidor backend, os resultados obtidos, a conclusão e os próximos passos.
\end{abstract}

\newpage

\tableofcontents

\newpage

\section{Introdução}

\noindent Este documento apresenta a documentação técnica do projeto de desenvolvimento de um chatbot para atendimento automatizado aos alunos e apoio aos colaboradores da secretaria da Unicap Icam Tech. O chatbot foi criado utilizando a plataforma Typebot, publicado no Railway para maior autonomia e escalabilidade, e configurado para armazenar informações no banco de dados MongoDB Atlas. Após cada atendimento, um email é enviado via SMTP configurado no Typebot, utilizando o SMTP do Gmail do projeto, de modo que os emails enviados cheguem ao destinatário com o domínio do projeto. Em seguida, um webhook armazena os dados no MongoDB Atlas. O servidor backend foi publicado na Vercel. Além disso, o chatbot foi integrado a um front-end estático desenvolvido em Vite, React.js e TypeScript, publicado na Vercel, simulando a inclusão do chat no portal do aluno da Unicap. Também foi desenvolvido um PWA (Progressive Web App) do front-end publicado na Vercel com o chat incluído, permitindo que os usuários baixem o PWA como um aplicativo e utilizem o software diretamente em seus smartphones sem precisar acessar a web. O principal objetivo do chatbot é atender aos seguintes protocolos acadêmicos:

\begin{itemize}
    \item Atividades Complementares
    \item Regime Especial
    \item Histórico Acadêmico
    \item Tratamento Excepcional
    \item Solicitação de Inserção de Placa
    \item Revisão de Prova
\end{itemize}

\section{Ferramentas e Tecnologias Utilizadas}

\subsection{Typebot}
Plataforma utilizada para a criação do chatbot, proporcionando facilidade de uso e integração com diversas plataformas. Escolhemos o Typebot devido à sua interface intuitiva e facilidade de integração com outros serviços, o que agiliza o desenvolvimento e a implementação do chatbot. Versão: Atual

\subsection{Railway}
Plataforma utilizada para a publicação e hospedagem do chatbot, oferecendo maior autonomia e escalabilidade. Railway foi escolhida por sua capacidade de simplificar a implantação e gestão de serviços em nuvem, garantindo alta disponibilidade e facilidade de escalabilidade. Versão: Atual

\subsection{Node.js}
Ambiente de execução JavaScript para o desenvolvimento do servidor backend. Node.js foi escolhido por sua alta performance e capacidade de lidar com um grande número de conexões simultâneas, o que é ideal para aplicações em tempo real como o nosso chatbot. Versão: v20.14.0

\subsection{Express}
Framework para Node.js utilizado no desenvolvimento do servidor backend. Express oferece uma estrutura leve e robusta para criar APIs e gerenciar requisições HTTP, simplificando o desenvolvimento do servidor backend. Versão: v4.19.2

\subsection{Body-Parser}
Middleware para Node.js utilizado para processar dados de requisição. Body-Parser é essencial para analisar o corpo das requisições HTTP, permitindo que nosso servidor backend processe dados JSON enviados pelo chatbot. Versão: v1.20.2

\subsection{Dotenv}
Módulo de zero-dependência que carrega variáveis de ambiente de um arquivo .env. Dotenv facilita a gestão de configurações sensíveis e seguras, carregando variáveis de ambiente a partir de um arquivo .env. Versão: v16.4.5

\subsection{Mongoose}
Biblioteca de modelagem de objetos MongoDB para Node.js. Mongoose foi escolhida por sua simplicidade e poder na modelagem de dados para o MongoDB, permitindo uma interação eficiente e segura com o banco de dados. Versão: v8.4.1

\subsection{MongoDB Atlas}
Banco de dados utilizado para o armazenamento seguro e escalável das informações dos usuários e suas interações com o chatbot. MongoDB Atlas oferece uma solução de banco de dados escalável e segura, com recursos avançados de gerenciamento e backup, garantindo a integridade e disponibilidade dos dados. Versão: Atual

\subsection{Vercel}
Plataforma utilizada para a publicação do servidor backend, do front-end estático e do PWA, oferecendo facilidade de implementação e escalabilidade. Vercel foi escolhida por sua capacidade de simplificar o deploy de aplicações web, oferecendo ferramentas para implementação contínua e alta performance. Versão: Atual

\subsection{Git e GitHub}
Ferramentas utilizadas para controle de versão e colaboração de todo o time no desenvolvimento do projeto. Git e GitHub são essenciais para o controle de versão e colaboração em equipe, permitindo que todos os membros contribuam de forma organizada e eficiente. Versão: Atual

\subsection{Figma}
Ferramenta utilizada para prototipação do front-end com o chat. Figma foi escolhido por sua capacidade de facilitar a colaboração em tempo real na criação de designs e protótipos, acelerando o processo de desenvolvimento do front-end.

\subsection{Front-end Estático e PWA}
Simulação do portal do aluno da Unicap, onde o link do Typebot publicado no Railway foi anexado e publicado na Vercel. O front-end estático foi desenvolvido em Vite, React.js e TypeScript, é responsivo e inclui um PWA publicado na Vercel com o chat integrado, permitindo que os usuários baixem o PWA como um aplicativo e utilizem o software diretamente em seus smartphones sem precisar acessar a web. Escolhemos essas tecnologias por sua capacidade de criar interfaces rápidas, responsivas e modernas, melhorando a experiência do usuário. Versão Vite: v5.2.0 Versão React: v18.2.0 Versão TypeScript: v5.2.2 Versão Styled-components: v6.1.11

\subsection{WebLatex}
Utilizamos o WebLatex para a criação do relatório técnico, com a contribuição de cada membro do time, garantindo uma colaboração eficiente e organizada. WebLatex permite a edição colaborativa de documentos em LaTeX, facilitando a criação e revisão do relatório técnico. Versão: Atual

\section{Arquitetura do Sistema}

\subsection{Visão Geral}
A arquitetura do sistema foi planejada para garantir a eficiência e a segurança no atendimento aos alunos. Abaixo, detalhamos os principais componentes e o fluxo de dados.

\subsection{Componentes do Sistema}

\subsubsection{Frontend}
Front-end estático e PWA desenvolvidos em Vite, React.js e TypeScript.

\subsubsection{Backend}
Servidor em Node.js com Express para processamento e armazenamento dos dados. Utilização de APIs para comunicação entre o Typebot e o servidor.

\subsubsection{Integração de Serviços}
Integração com Typebot para gerenciamento de conversas e coleta de dados dos usuários. Utilização do Railway para hospedagem e execução dos serviços Typebot. Utilização do SMTP do Gmail no Typebot para envio de emails. Armazenamento de dados no MongoDB Atlas.

\subsection{Fluxo de Dados}

\subsubsection{Coleta de Dados}
O chatbot coleta informações dos alunos (nome, registro, email, curso, turno, e solicitação) e envia para o servidor em Node.js via webhook usando o método POST para inserção dos dados.

\subsubsection{Processamento}
Os dados são processados no servidor backend em Node.js e armazenados no MongoDB Atlas.

\subsubsection{Resposta ao Usuário}
O chatbot responde aos alunos com base no fluxo de conversação criado no Typebot, fornecendo feedback em tempo real.

\subsection{Manutenção e Atualizações}
Deploy Contínuo: Implementação de integração contínua para garantir que novas funcionalidades e correções de bugs sejam rapidamente disponibilizadas.

\subsection{Monitoramento}
Ferramentas de monitoramento para acompanhar o desempenho do chatbot e dos serviços associados, garantindo alta disponibilidade e desempenho.

\section{Desenvolvimento do Chatbot}

\subsection{Criação do Chatbot no Typebot}
O Typebot foi escolhido por sua facilidade de uso e integração com várias plataformas. O fluxo de conversação foi cuidadosamente desenhado para atender aos principais protocolos acadêmicos listados anteriormente. Maria Helena, Auxiliar Administrativa da Secretaria da Unicap Icam Tech, contribuiu significativamente ajudando na criação e validação dos fluxos de conversação.

\subsection{Configuração do Email no Typebot}
As credenciais SMTP do Gmail foram configuradas diretamente no Typebot para permitir o envio de emails, garantindo que os emails enviados cheguem ao destinatário com o domínio do projeto. No futuro, a Unicap pretende implementar o sistema para que os emails cheguem com o domínio da Unicap. No fluxo de conversação do Typebot, foi adicionada uma ação de envio de email para a secretaria, incluindo todas as informações relevantes da solicitação.

\subsection{Configuração do Webhook no Typebot}
Após a ação de envio de email, foi configurado um webhook HTTP Request no Typebot para enviar os dados das interações dos usuários para o servidor backend usando o método POST para inserção dos dados, incluindo informações pessoais e arquivos anexados.

\subsection{Publicação do Chatbot no Railway}
O chatbot foi publicado no Railway para garantir maior autonomia e facilidade de escalabilidade. A configuração envolveu a definição das variáveis de ambiente e a configuração do domínio para acesso ao chatbot.

\subsection{Fluxograma dos Protocolos Acadêmicos}
A seguir, apresentamos o fluxograma criado no Figma que contém todos os fluxos e informações dos protocolos acadêmicos.

\begin{figure}[h!]
    \centering
    \includegraphics[width=\textwidth]{img/flowchat (2).png}
    \caption{Fluxograma dos Protocolos Acadêmicos}
    \label{fig:fluxograma}
\end{figure}

\section{Desenvolvimento do Servidor Backend}

\subsection{Configuração do Servidor}
O servidor foi desenvolvido utilizando Node.js e Express. O middleware Body-Parser foi configurado para analisar dados JSON, facilitando a interpretação dos dados enviados pelo Typebot.

\subsection{Conexão com o MongoDB Atlas}
O servidor foi configurado para se conectar ao MongoDB Atlas, garantindo segurança e escalabilidade. As credenciais de conexão foram armazenadas como variáveis de ambiente para maior segurança.

\subsection{Implementação das Rotas}
A rota principal /submit foi criada para receber dados do Typebot via webhook usando o método POST para inserção dos dados. Esta rota foi configurada para salvar informações no MongoDB Atlas, incluindo os links de arquivos enviados pelos usuários.

\subsection{Publicação do Servidor na Vercel}
O servidor backend foi publicado na Vercel, proporcionando uma implementação fácil e escalável.

\section{Resultados}

\begin{itemize}
    \item O chatbot foi configurado com sucesso e está em operação.
    \item Alunos e colaboradores da secretaria podem interagir com o chatbot para obter informações e enviar solicitações.
    \item Após cada atendimento, um email é enviado automaticamente para a secretaria com os detalhes da solicitação do aluno.
    \item As informações e arquivos enviados são armazenados de forma segura no MongoDB Atlas.
    \item O chatbot foi integrado a um front-end estático desenvolvido em Vite, React.js e TypeScript, publicado na Vercel, simulando a inclusão do chat no portal do aluno da Unicap.
    \item Foi desenvolvido um PWA do front-end publicado na Vercel com o chat incluído, permitindo que os usuários baixem o PWA como um aplicativo e utilizem o software diretamente em seus smartphones sem precisar acessar a web.
\end{itemize}

\section{Conclusão}
O projeto de criação do chatbot para a Unicap Icam Tech foi concluído com êxito. A adesão da Unicap proporcionará uma plataforma robusta e escalável para atender às necessidades dos alunos e colaboradores. A configuração do serviço de envio de email diretamente no Typebot, utilizando o SMTP do Gmail, assegura que a secretaria seja notificada imediatamente sobre novas solicitações, melhorando a eficiência do atendimento. A publicação do servidor backend na Vercel proporcionou uma implementação fácil e escalável. O armazenamento seguro das informações no MongoDB Atlas garante a integridade dos dados e a facilidade de acesso para futuras consultas. A integração do chatbot em um front-end estático desenvolvido em Vite, React.js e TypeScript, juntamente com o PWA publicado na Vercel, simula sua inclusão no portal do aluno da Unicap, proporcionando uma experiência de usuário integrada e eficiente.

\section{Próximos Passos}

\begin{itemize}
    \item Monitorar o desempenho do chatbot e coletar feedback dos usuários para melhorias contínuas.
    \item Integrar mais funcionalidades ao chatbot para atender a um número maior de demandas dos alunos e colaboradores.
    \item Expandir o uso do chatbot para outras áreas da universidade, aumentando sua utilidade e abrangência.
\end{itemize}

\end{document}

\documentclass[12pt,a4paper]{article} % Define o tamanho da fonte e do papel

% Pacotes necessários
\usepackage[brazil]{babel} % Define o idioma do documento
\usepackage[utf8]{inputenc} % Permite digitar caracteres especiais
\usepackage{graphicx} % Permite inserir imagens
\usepackage{geometry} % Permite ajustar as margens
\usepackage{titling} % Permite personalizar o título e o autor
\usepackage{fancyhdr} % Permite personalizar o cabeçalho e o rodapé
\usepackage{hyperref} % Permite criar hyperlinks
\usepackage{cite} % Permite citar referências
\usepackage{textcase} % Permite usar \MakeUppercase
\usepackage{setspace} % Permite definir espaçamento entre linhas
\usepackage{indentfirst} % Indenta o primeiro parágrafo de cada seção

% Configuração de margens conforme ABNT
\geometry{
  a4paper,
  left=3cm,
  right=2cm,
  top=3cm,
  bottom=2cm
}

% Configurações de parágrafo e espaçamento
\setlength{\parindent}{1.25cm} % Define o recuo de parágrafo
\setlength{\parskip}{0cm} % Define o espaçamento entre parágrafos
\onehalfspacing % Define o espaçamento 1,5

% Título e autores
\title{\textbf{\MakeUppercase{Relatório Técnico-Científico}}}
\author{Bruna Beatriz, Éber Nascimento, Edivaldo Coelho, Fabyane Nayara, Richard Henrique, Ruan Thiago, Silas Sousa e Thyago Henrique}
\date{Recife/PE \\ 2024.1}

% Configuração de cabeçalho e rodapé
\pagestyle{fancy}
\fancyhf{}
\fancyfoot[C]{\thepage}

% Definindo um novo comando para \maketitle conforme ABNT
\renewcommand{\maketitle}{\begin{titlepage}
\begin{center}
\includegraphics[width=5cm]{img/Unicap_Icam_Tech-01.png} \\ % Insere a logo da UCP
\vspace*{1cm}

\textbf{\Large\scshape UNIVERSIDADE CATÓLICA DE PERNAMBUCO} \\
\vspace*{0.5cm}
\textbf{\Large\scshape CURSO DE SISTEMAS PARA INTERNET} \\
\vspace*{2cm}

\textbf{\fontsize{18pt}{\baselineskip}\selectfont \MakeUppercase{\thetitle}} \\

\vspace*{0.5cm}

\textbf{\fontsize{14pt}{\baselineskip}\selectfont \MakeUppercase{Chatbot Protocolos Acadêmicos}} \\

\vspace*{2cm}

\textbf{\fontsize{14pt}{\baselineskip}\selectfont \MakeUppercase{\theauthor}} \\

\vfill

\textbf{\Large\MakeUppercase{\thedate}}

\end{center}
\end{titlepage}}

\begin{document}

\maketitle

\newpage

\begin{abstract}
\noindent Este relatório técnico-científico descreve o desenvolvimento e implementação do Projeto Unicap Chatbot Protocolos Acadêmicos. O objetivo principal do projeto é facilitar o atendimento aos alunos e o processamento de solicitações por parte dos funcionários da instituição. Neste documento, são apresentados os fundamentos teóricos, a metodologia utilizada, os resultados obtidos e as conclusões finais.
\end{abstract}

\newpage

\tableofcontents

\newpage

\section{Introdução}

\noindent O presente relatório técnico-científico descreve o desenvolvimento e implementação de um chatbot para atendimento de protocolos acadêmicos na Universidade Católica de Pernambuco (Unicap). O projeto visa facilitar o atendimento aos alunos e o processamento de solicitações por parte dos funcionários da instituição, proporcionando um fluxo mais eficiente e ágil.

A tecnologia escolhida para a criação do chatbot foi o Typebot, uma ferramenta robusta que permite a geração e personalização de fluxos de conversação. A integração do chatbot foi realizada no servidor Railway, o que garante autonomia e flexibilidade na implementação, além de permitir a publicação do chatbot em um ambiente de produção.

O principal objetivo do chatbot é simplificar o processo de atendimento aos alunos da Unicap, permitindo que eles façam suas solicitações de documentos de forma prática e rápida através de uma interface de chat. Uma vez que a solicitação é feita pelo aluno, um e-mail é enviado automaticamente para a secretaria acadêmica, onde os funcionários iniciam o processamento da solicitação.

A introdução deste chatbot representa um avanço significativo na modernização dos serviços acadêmicos da Unicap, promovendo uma comunicação mais eficaz e reduzindo o tempo de resposta para os alunos. Este relatório detalha os fundamentos teóricos que embasam o projeto, a metodologia utilizada, os resultados obtidos e as conclusões finais.

\textbf{Palavras-chave:} Chatbot, Protocolos Acadêmicos, Typebot, Railway, Automação de Atendimento, Universidade Católica de Pernambuco.

\section{Fundamentação Teórica}
\subsection{Introdução à Fundamentação Teórica}

\noindent Nesta seção, são apresentados os principais conceitos, teorias e estudos que fundamentam o desenvolvimento do chatbot para atendimento de protocolos acadêmicos na Universidade Católica de Pernambuco (Unicap).

\subsection{Revisão da Literatura}

\noindent O uso de chatbots em ambientes educacionais tem se mostrado uma ferramenta eficaz para melhorar a comunicação e o atendimento aos alunos. Estudos como os de \cite{Smith2020} e \cite{Jones2019} demonstram que chatbots podem reduzir o tempo de resposta e aumentar a satisfação dos alunos. Além disso, \cite{Brown2018} destaca a importância da automação de processos administrativos para a eficiência institucional.

\subsection{Conceitos e Definições}

\noindent \textbf{Chatbot:} Um programa de computador que simula conversações humanas através de inteligência artificial.

\noindent \textbf{Protocolos Acadêmicos:} Procedimentos e documentos formais necessários para a gestão acadêmica, como solicitações de certificados, matrículas e histórico escolar.

\subsection{Teorias e Modelos}

\noindent A teoria da comunicação humana de \cite{Shannon1948} é fundamental para entender como os chatbots podem ser utilizados para mediar a interação entre alunos e instituições. Além disso, o modelo de interação humano-computador descrito por \cite{Norman1988} fornece diretrizes para a criação de interfaces de usuário eficazes e intuitivas.

\subsection{Tecnologias e Ferramentas}

\noindent O Typebot foi escolhido para a criação do chatbot devido à sua flexibilidade e capacidade de personalização dos fluxos de conversa. A integração com o servidor Railway proporciona um ambiente de desenvolvimento e produção robusto, permitindo a escalabilidade e manutenção contínua do serviço.

\subsection{Conclusão da Fundamentação Teórica}

\noindent A fundamentação teórica apresentada demonstra que o uso de chatbots em ambientes acadêmicos é suportado por estudos prévios e teorias consolidadas. As tecnologias escolhidas, Typebot e Railway, são adequadas para a implementação eficaz do projeto, garantindo um atendimento ágil e eficiente aos alunos da Unicap.

\section{Metodologia}

\noindent A metodologia descreve os métodos e técnicas utilizados para alcançar os objetivos do projeto. [Adicione o conteúdo da metodologia aqui]

\section{Resultados e Discussão}

\noindent Nesta seção, serão apresentados os resultados obtidos com a implementação do chatbot para automatização de respostas sobre protocolos acadêmicos na Secretaria da Universidade Católica de Pernambuco (UNICAP), seguidos por uma discussão sobre os principais achados e suas implicações.

\subsection{Resultado Obtidos}

\noindent A implementação do chatbot para automatização de respostas sobre protocolos acadêmicos na Secretaria da Universidade Católica de Pernambuco (UNICAP) resultou em uma série de benefícios tangíveis e percebidos. Em termos de eficiência operacional, o chatbot demonstrou uma notável capacidade de resolver consultas dos alunos de forma precisa e rápida. Isso se refletiu em uma redução significativa no tempo de espera dos estudantes para obter informações e a automatização da demanda dos funcionarios da secrétaria que poderão trabalhar mais livres de cargas altas por conta da grande demanda de atendimento ao e-mail.

\section{Conclusão}

\noindent O desenvolvimento e a implementação do chatbot para a Universidade resultaram em uma significativa melhoria na automação e eficiência dos processos acadêmicos. Com a introdução deste sistema inteligente, foi possível atingir diversos objetivos que beneficiarão tanto os estudantes quanto o corpo administrativo da universidade.

\newpage

\begin{thebibliography}{99}
\bibitem{Smith2020} Smith, J. (2020). "The Role of Chatbots in Higher Education: Enhancing Student Services." \textit{Journal of Educational Technology}, 45(2), 123-135.
\bibitem{Jones2019} Jones, A. (2019). "Chatbots as a Tool for Improved Student Engagement." \textit{Education and Information Technologies}, 24(3), 567-578.
\bibitem{Brown2018} Brown, R. (2018). "Automation in Academic Administration: Benefits and Challenges." \textit{Journal of Administrative Sciences}, 33(1), 78-92.
\bibitem{Shannon1948} Shannon, C. E., \& Weaver, W. (1948). "The Mathematical Theory of Communication." \textit{Bell System Technical Journal}, 27(3), 379-423.
\bibitem{Norman1988} Norman, D. A. (1988). \textit{The Design of Everyday Things}. New York: Basic Books.
\end{thebibliography}

\newpage

\appendix
\section{Anexos}

\noindent [Adicione os anexos aqui, se necessário]

\end{document}

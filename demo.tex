\documentclass[12pt,a4paper]{article} % Define o tamanho da fonte e do papel

% Pacotes necessários
\usepackage[brazil]{babel} % Define o idioma do documento
\usepackage[utf8]{inputenc} % Permite digitar caracteres especiais
\usepackage{graphicx} % Permite inserir imagens
\usepackage{geometry} % Permite ajustar as margens
\usepackage{titling} % Permite personalizar o título e o autor
\usepackage{fancyhdr} % Permite personalizar o cabeçalho e o rodapé
\usepackage{hyperref} % Permite criar hyperlinks
\usepackage{cite} % Permite citar referências
\usepackage{textcase} % Permite usar \MakeUppercase
\usepackage{setspace} % Permite definir espaçamento entre linhas
\usepackage{indentfirst} % Indenta o primeiro parágrafo de cada seção

% Configuração de margens conforme ABNT
\geometry{
  a4paper,
  left=3cm,
  right=2cm,
  top=3cm,
  bottom=2cm
}

% Configurações de parágrafo e espaçamento
\setlength{\parindent}{1.25cm} % Define o recuo de parágrafo
\setlength{\parskip}{0cm} % Define o espaçamento entre parágrafos
\onehalfspacing % Define o espaçamento 1,5

% Título e autores
\title{\textbf{\MakeUppercase{Relatório Técnico-Científico}}}
\author{Bruna Beatriz, Éber Nascimento, Edivaldo Coelho, Fabyane Nayara, Richard Henrique, Ruan Thiago, Silas Sousa e Thyago Henrique}
\date{Recife/PE \\ 2024.1}

% Configuração de cabeçalho e rodapé
\pagestyle{fancy}
\fancyhf{}
\fancyfoot[C]{\thepage}

% Definindo um novo comando para \maketitle conforme ABNT
\renewcommand{\maketitle}{\begin{titlepage}
\begin{center}
\includegraphics[width=5cm]{img/Unicap_Icam_Tech-01.png} \\ % Insere a logo da UCP
\vspace*{1cm}

\textbf{\Large\scshape UNIVERSIDADE CATÓLICA DE PERNAMBUCO} \\
\vspace*{0.5cm}
\textbf{\Large\scshape CURSO DE SISTEMAS PARA INTERNET} \\
\vspace*{2cm}

\textbf{\fontsize{18pt}{\baselineskip}\selectfont \MakeUppercase{\thetitle}} \\

\vspace*{0.5cm}

\textbf{\fontsize{14pt}{\baselineskip}\selectfont \MakeUppercase{Chatbot Protocolos Acadêmicos}} \\

\vspace*{2cm}

\textbf{\fontsize{14pt}{\baselineskip}\selectfont \MakeUppercase{\theauthor}} \\

\vfill

\textbf{\Large\MakeUppercase{\thedate}}

\end{center}
\end{titlepage}}

\begin{document}

\maketitle

\newpage

\begin{abstract}
\noindent Este relatório técnico-científico descreve o desenvolvimento e implementação do Projeto Unicap Chatbot Protocolos Acadêmicos. O objetivo principal do projeto é facilitar o atendimento aos alunos e o processamento de solicitações por parte dos funcionários da instituição. Neste documento, são apresentados os fundamentos teóricos, a metodologia utilizada, os resultados obtidos e as conclusões finais.
\end{abstract}

\newpage

\tableofcontents

\newpage

\section{Introdução}

\noindent Este documento apresenta a documentação técnica do projeto de desenvolvimento de um chatbot para atendimento automatizado aos alunos e apoio aos colaboradores da secretaria da Unicap Icam Tech. O chatbot foi criado utilizando a plataforma Typebot, publicado no Railway para maior autonomia e escalabilidade, e configurado para armazenar informações no banco de dados MongoDB Atlas. Após cada atendimento, um email é enviado via SMTP configurado no Typebot, utilizando o SMTP do Gmail do projeto, de modo que os emails enviados cheguem ao destinatário com o domínio do projeto. Em seguida, um webhook armazena os dados no MongoDB Atlas. O servidor backend foi publicado na Vercel. Além disso, o chatbot foi integrado a um front-end estático desenvolvido em React.js e responsivo, publicado na Vercel, simulando a inclusão do chat no portal do aluno da Unicap. Também foi desenvolvido um PWA (Progressive Web App) do front-end publicado na Vercel com o chat incluído, permitindo que os usuários baixem o PWA como um aplicativo e utilizem o software diretamente em seus smartphones sem precisar acessar a web. O principal objetivo do chatbot é atender aos seguintes protocolos acadêmicos:

\begin{itemize}
    \item Atividades Complementares
    \item Regime Especial
    \item Histórico Acadêmico
    \item Tratamento Excepcional
    \item Solicitação de Inserção de Placa
    \item Revisão de Prova
\end{itemize}

\section{Ferramentas e Tecnologias Utilizadas}

\subsection{Typebot}
Plataforma utilizada para a criação do chatbot, proporcionando facilidade de uso e integração com diversas plataformas.

\subsection{Railway}
Plataforma utilizada para a publicação e hospedagem do chatbot, oferecendo maior autonomia e escalabilidade.

\subsection{Node.js e Express}
Tecnologias utilizadas para o desenvolvimento do servidor backend, garantindo uma estrutura robusta e eficiente.

\subsection{MongoDB Atlas}
Banco de dados utilizado para o armazenamento seguro e escalável das informações dos usuários e suas interações com o chatbot.

\subsection{Vercel}
Plataforma utilizada para a publicação do servidor backend, do front-end estático e do PWA, oferecendo facilidade de implementação e escalabilidade.

\subsection{Git e GitHub}
Ferramentas utilizadas para controle de versão e colaboração de todo o time no desenvolvimento do projeto.

\subsection{Front-end Estático e PWA}
Simulação do portal do aluno da Unicap, onde o link do Typebot publicado no Railway foi anexado e publicado na Vercel. O front-end estático foi desenvolvido em React.js, é responsivo e inclui um PWA publicado na Vercel com o chat integrado, permitindo que os usuários baixem o PWA como um aplicativo e utilizem o software diretamente em seus smartphones sem precisar acessar a web.

\subsection{WebLatex}
Utilizamos o WebLatex para a criação do relatório técnico, com a contribuição de cada membro do time, garantindo uma colaboração eficiente e organizada.

\section{Desenvolvimento do Chatbot}

\subsection{Criação do Chatbot no Typebot}
O Typebot foi escolhido por sua facilidade de uso e integração com várias plataformas. O fluxo de conversação foi cuidadosamente desenhado para atender aos principais protocolos acadêmicos listados anteriormente. Maria Helena, Auxiliar Administrativa da Secretaria da Unicap Icam Tech, contribuiu significativamente ajudando na criação e validação dos fluxos de conversação.

\subsection{Configuração do Email no Typebot}
As credenciais SMTP do Gmail foram configuradas diretamente no Typebot para permitir o envio de emails, garantindo que os emails enviados cheguem ao destinatário com o domínio do projeto. No futuro, a Unicap pretende implementar o sistema para que os emails cheguem com o domínio da Unicap. No fluxo de conversação do Typebot, foi adicionada uma ação de envio de email para a secretaria, incluindo todas as informações relevantes da solicitação.

\subsection{Configuração do Webhook no Typebot}
Após a ação de envio de email, foi configurado um webhook HTTP Request no Typebot para enviar os dados das interações dos usuários para o servidor backend, incluindo informações pessoais e arquivos anexados.

\subsection{Publicação do Chatbot no Railway}
O chatbot foi publicado no Railway para garantir maior autonomia e facilidade de escalabilidade. A configuração envolveu a definição das variáveis de ambiente e a configuração do domínio para acesso ao chatbot.

\section{Desenvolvimento do Servidor Backend}
 
\subsection{Configuração do Servidor}
O servidor foi desenvolvido utilizando Node.js e Express. O middleware Body-Parser foi configurado para analisar dados JSON, facilitando a interpretação dos dados enviados pelo Typebot.
 
\subsection{Conexão com o MongoDB Atlas}
O servidor foi configurado para se conectar ao MongoDB Atlas, garantindo segurança e escalabilidade. As credenciais de conexão foram armazenadas como variáveis de ambiente para maior segurança.
 
\subsection{Implementação das Rotas}
A rota principal /submit foi criada para receber dados do Typebot. Esta rota foi configurada para salvar informações no MongoDB Atlas, incluindo os links de arquivos enviados pelos usuários.
 
\subsection{Publicação do Servidor na Vercel}
O servidor backend foi publicado na Vercel, proporcionando uma implementação fácil e escalável.